% LaTeX template for academic reports (thesis)

\documentclass[12pt,english,a4paper,oneside]{article}
\let\circledS\undefined
\usepackage{setspace}
\setstretch{2.0}
\usepackage{amssymb,amsmath}
\usepackage{ifxetex,ifluatex}
\usepackage[nottoc]{tocbibind}
\usepackage{csquotes}
\usepackage[usenames,dvipsnames]{xcolor}
\usepackage[bookmarks, colorlinks, breaklinks]{hyperref}
\definecolor{mannheimblue}{HTML}{003056}
\definecolor{mannheimorange}{HTML}{df7e50}

\hypersetup{linkcolor=mannheimblue,
citecolor=mannheimblue,
filecolor=black,
urlcolor=mannheimblue}


% some more packages...
\usepackage{graphicx}
%\usepackage{scrpage2}
%\usepackage{xcolor}
\usepackage{hyperref}
%\hypersetup{colorlinks=true, linkcolor = blue, urlcolor = blue}
%\usepackage{eso-pic}

\renewenvironment{quote}{\list{}\item\relax
\small\singlespacing}{\endlist}
\SetBlockEnvironment{quote}

\onehalfspacing
% \renewcommand{\baselinestretch}{1.5}  % line distance is 1.5

%\renewcommand{\chaptername}{} %% remove the word \chapter

% % \newlength{\cslhangindent}
% \setlength{\cslhangindent}{1.5em}
% \newenvironment{CSLReferences}%
%   {\setlength{\parindent}{0pt}%
%   \everypar{\setlength{\hangindent}{\cslhangindent}}\ignorespaces}%
%   {\par}
% 
% Pandoc citation processing
\newlength{\cslhangindent}
\setlength{\cslhangindent}{1.5em}
\newlength{\csllabelwidth}
\setlength{\csllabelwidth}{3em}
\newlength{\cslentryspacingunit} % times entry-spacing
\setlength{\cslentryspacingunit}{\parskip}
% for Pandoc 2.8 to 2.10.1
\newenvironment{cslreferences}%
  {\setlength{\parindent}{0pt}%
  \everypar{\setlength{\hangindent}{\cslhangindent}}\ignorespaces}%
  {\par}
% For Pandoc 2.11+
\newenvironment{CSLReferences}[2] % #1 hanging-ident, #2 entry spacing
 {% don't indent paragraphs
  \setlength{\parindent}{0pt}
  % turn on hanging indent if param 1 is 1
  \ifodd #1
  \let\oldpar\par
  \def\par{\hangindent=\cslhangindent\oldpar}
  \fi
  % set entry spacing
  \setlength{\parskip}{#2\cslentryspacingunit}
 }%
 {}
\usepackage{calc}
\newcommand{\CSLBlock}[1]{#1\hfill\break}
\newcommand{\CSLLeftMargin}[1]{\parbox[t]{\csllabelwidth}{#1}}
\newcommand{\CSLRightInline}[1]{\parbox[t]{\linewidth - \csllabelwidth}{#1}\break}
\newcommand{\CSLIndent}[1]{\hspace{\cslhangindent}#1}
% % \newlength{\cslhangindent}
% \setlength{\cslhangindent}{1.5em}
% \newlength{\csllabelwidth}
% \setlength{\csllabelwidth}{3em}
% \newenvironment{CSLReferences}[3] % #1 hanging-ident, #2 entry spacing
%  {% don't indent paragraphs
%   \setlength{\parindent}{0pt}
%   % turn on hanging indent if param 1 is 1
%   \ifodd #1 \everypar{\setlength{\hangindent}{\cslhangindent}}\ignorespaces\fi
%   % set entry spacing
%   \ifnum #2 > 0
%   \setlength{\parskip}{#2\baselineskip}
%   \fi
%  }%
%  {}
% \usepackage{calc} % for \widthof, \maxof
% \newcommand{\CSLBlock}[1]{#1\hfill\break}
% \newcommand{\CSLLeftMargin}[1]{\parbox[t]{\maxof{\widthof{#1}}{\csllabelwidth}}{#1}}
% \newcommand{\CSLRightInline}[1]{\parbox[t]{\linewidth}{#1}}
% \newcommand{\CSLIndent}[1]{\hspace{\cslhangindent}#1}
% 
\usepackage{fixltx2e} % provides \textsubscript
\ifnum 0\ifxetex 1\fi\ifluatex 1\fi=0 % if pdftex
  \usepackage[T1]{fontenc}
  \usepackage[utf8]{inputenc}
\else % if luatex or xelatex
  \ifxetex
    \usepackage{mathspec}
  \else
    \usepackage{fontspec}
  \fi
  \defaultfontfeatures{Ligatures=TeX,Scale=MatchLowercase}

\fi
% % use upquote if available, for straight quotes in verbatim environments
% \IfFileExists{upquote.sty}{\usepackage{upquote}}{}
% % use microtype if available
% \IfFileExists{microtype.sty}{%
% \usepackage{microtype}
% \UseMicrotypeSet[protrusion]{basicmath} % disable protrusion for tt fonts
% }{}
% % \usepackage[left=2.5cm,right=2.5cm,top=2.5cm,bottom=2.5cm]{geometry}
% \usepackage{hyperref}
\hypersetup{unicode=true,
            pdftitle={On Using the Metropolis-Hastings Algorithm for Data Imputation},
            pdfauthor={Tobias Stenzel},
            pdfborder={0 0 0},
            breaklinks=true}
\urlstyle{same}  % don't use monospace font for urls
\ifnum 0\ifxetex 1\fi\ifluatex 1\fi=0 % if pdftex
  \usepackage[shorthands=off,main=english]{babel}
\else
  \usepackage{polyglossia}
  \setmainlanguage[]{english}
\fi
% % \usepackage{longtable,booktabs}
\setlength{\emergencystretch}{3em}  % prevent overfull lines
\providecommand{\tightlist}{%
  \setlength{\itemsep}{0pt}\setlength{\parskip}{0pt}}
\setcounter{secnumdepth}{5}
% Redefines (sub)paragraphs to behave more like sections
\ifx\paragraph\undefined\else
\let\oldparagraph\paragraph
\renewcommand{\paragraph}[1]{\oldparagraph{#1}\mbox{}}
\fi
\ifx\subparagraph\undefined\else
\let\oldsubparagraph\subparagraph
\renewcommand{\subparagraph}[1]{\oldsubparagraph{#1}\mbox{}}
\fi


\usepackage{csquotes}
\usepackage{fancyhdr} % to change header and footers
\usepackage{url}
\def\UrlBreaks{\do\/\do-}

%%%% plagiarism

\newcommand*{\SignatureAndDate}[1]{%
\vspace{2cm}
     Mannheim, den \makebox[2cm]{\hrulefill} \hfill\makebox[9cm]{\hrulefill}%
     \par
%  \makebox[2cm]{ Ort, Datum}
  \hfill\makebox[7.5cm][t]{Name und Unterschrift}
  \vspace{2cm}
}%

 \newcommand*{\SignatureAndDateEng}[1]{%
\vspace{2cm}
     Mannheim, \makebox[2cm]{\hrulefill} \hfill\makebox[9cm]{\hrulefill}%
     \par
    \hfill\makebox[7.5cm][t]{Name and Signature}%
\vspace{2cm}
}%


\makeatletter
\newenvironment{kframe}{%
\medskip{}
\setlength{\fboxsep}{.8em}
 \def\at@end@of@kframe{}%
 \ifinner\ifhmode%
  \def\at@end@of@kframe{\end{minipage}}%
  \begin{minipage}{\columnwidth}%
 \fi\fi%
 \def\FrameCommand##1{\hskip\@totalleftmargin \hskip-\fboxsep
 \colorbox{shadecolor}{##1}\hskip-\fboxsep
     % There is no \\@totalrightmargin, so:
     \hskip-\linewidth \hskip-\@totalleftmargin \hskip\columnwidth}%
 \MakeFramed {\advance\hsize-\width
   \@totalleftmargin\z@ \linewidth\hsize
   \@setminipage}}%
 {\par\unskip\endMakeFramed%
 \at@end@of@kframe}
\makeatother

%\renewenvironment{Shaded}{\begin{kframe}}{\end{kframe}} ses 2019-03-08












\newcommand{\ts}{\thinspace}



\usepackage{tikz, float, caption, amsthm, algorithm, algpseudocode}
\floatplacement{figure}{H}
\interfootnotelinepenalty=10000

%% font EB Garamond
\setmainfont[
Path = fonts/static/,
BoldFont = EBGaramond-Bold.ttf,
ItalicFont = EBGaramond-Italic.ttf,
BoldItalicFont  = EBGaramond-BoldItalic.ttf]
{EBGaramond-Regular.ttf}


% \usepackage{float}

\usepackage{amsthm}
\newtheorem{theorem}{Theorem}[section]
\newtheorem{lemma}{Lemma}[section]
\newtheorem{corollary}{Corollary}[section]
\newtheorem{proposition}{Proposition}[section]
\newtheorem{conjecture}{Conjecture}[section]
\theoremstyle{definition}
\newtheorem{definition}{Definition}[section]
\theoremstyle{definition}
\newtheorem{example}{Example}[section]
\theoremstyle{definition}
\newtheorem{exercise}{Exercise}[section]
\theoremstyle{definition}
\newtheorem{hypothesis}{Hypothesis}[section]
\theoremstyle{remark}
\newtheorem*{remark}{Remark}
\newtheorem*{solution}{Solution}
\begin{document}  %%%%%%% main document %%%%%%%%%%%%%%%
\begin{titlepage}

    \begin{center}
    \large{ \textsc{ \uppercase{University of Mannheim} \\ \vspace{-0.2cm}
School of Social Sciences \\ \vspace{-0.2cm}
Department of Political Science}}

      
        \vspace{3.5cm}
        

       \large{   Final Paper for Course   }


       \large{ \textit{   Advanced Quantitative Methods in Political Science   }}

\renewcommand{\linethickness}{0.03em}
\rule{\linewidth}{\linethickness}


       \LARGE{ \textbf{   On Using the Metropolis-Hastings Algorithm for Data Imputation   }}

        % \vspace{-0.5cm}

       \large{  }

        \vspace{-0.2cm}
\rule{\linewidth}{\linethickness}


\begin{minipage}[t]{0.5\textwidth}
\begin{flushleft}
\singlespacing
 \textbf{Tobias Stenzel}  \\ 


 \href{mailto:tobias.stenzel@students.uni-mannheim.de}{\nolinkurl{tobias.stenzel@students.uni-mannheim.de}}  \\ 

\end{flushleft}
\end{minipage}
\begin{minipage}[t]{0.4\textwidth}
\hfill
\end{minipage}\\
\vspace{0.2cm}
\begin{minipage}[t]{0.35\textwidth}
\hfill
\end{minipage}
\begin{minipage}[t]{0.55\textwidth}
\begin{flushright}
\singlespacing
     Prof.~Thomas Gschwend, Ph.D.  \\       

\end{flushright}
\end{minipage}\\
%


         \vfill
         Submission Date: Mai 12, 2022 \\ 
        





         \vfill



     \end{center}
    \thispagestyle{empty}
\end{titlepage}

\newpage
% \thispagestyle{empty}
% \mbox{}







{
\setcounter{tocdepth}{2}
\newpage
\pagenumbering{gobble}
\tableofcontents
}

\newpage
\pagenumbering{arabic}
\fancypagestyle{plain}{%
    \renewcommand{\headrulewidth}{0pt}%
    \fancyhf{}%
    \fancyfoot[R]{\thepage}%
}
% Set the right side of the footer to be the page number
\pagestyle{plain}
\hypertarget{introduction}{%
\section{Introduction}\label{introduction}}

In recent years scholars have found declining support for democracy in long-established democracies. (Denemark et al. 2016; Foa and Mounk 2016, 2017; Norris 2017; Voeten 2016). This finding raises two important questions: First, what are the reasons for this decline, and second, what are its implications, i.e., does a decline in democratic support endanger the survival of these democracies?

A series of recent articles (Claassen 2019, 2020a, 2020b) researches these questions and presents novel results. Regarding the first question, Claassen (2020a) finds -- in contrary to the the widely held belief of self-reinforcing democracies -- that democratic support naturally fluctuates over time. The reasons is that increases in democracy levels lead to decreases in public support and vice versa. Regarding question two, so far the results about the relationship between democratic support and its survival have been mixed. Claassen (2020a), however, finds supporting evidence for the natural theory that democratic support plays a positive role in the system's survival.

A main obstacle for researching democratic support is that current panel data contains a large number of missing values. Claassen (2019) finds an approach to simulate dense panel data from the actual fractured data. The two most recent studies both rely on this data. His approach consists of three steps: First, assume a probabilistic structural model of democratic support. Second, estimate its parameters via the Metropolis Hastings algorithm using the fractured data. Third, simulate the data using the model and the large number of parameter estimates.

The objective of this work is to evaluate the robustness of Claassen's findings towards changes in the method's hyperparameters. In other words, I incorparate the uncertainty about the right choice of hyperparameters into Claassen's results to test their reliability. In particular, I propagate a random set of hyperparameters through Claassen's estimation procedure. This is important because arbitrary hyperparameter choices can potentially lead to false or at least random results. Therefore, the respective parametric uncertainty should always be reported along the point estimate.

I come to the following results:

The final paper is structured as follows: Section 2 reviews the literature of public support with a focus on Claassen's papers and defines the concept. Section 3 explains Claassen's model of public support, its estimation and my approach for the uncertainty propagation. Section 4 presents the results and Section 5 discusses the findings. Section 6 concludes.

\hypertarget{literature-review}{%
\section{Literature Review}\label{literature-review}}

\hypertarget{the-concept-of-democratic-support}{%
\subsection{The Concept of Democratic Support}\label{the-concept-of-democratic-support}}

There are two major conceptualizations of public support for democracy (PSD). First, the \enquote{implicit} approach that requires the support for broader sociopolitical values like post-materialism and egalitarianism. Here, people support democracy implicitly if they support the values that are framed as particularly democratic. Second, the \enquote{explicit} approach that requires both an appraisal of democracy and a rejection of autocratic alternatives. Some studies also refer to PSD measured on the national level as \enquote{mood.}

\hypertarget{drivers-of-democratic-support}{%
\subsection{Drivers of Democratic Support}\label{drivers-of-democratic-support}}

The main theory about why citizen and societies begin to support democracy are called \emph{generational socialization} and \emph{instrumental regime performance}. The first theory assumes that individuals are taught to support the regime under which they are socialized during late adolescence (insert citation). One implication is that after a shift to democracy, the support for it will grow over time (insert citation). Indeed, several single-country studies have found evidence for this claim, for example for 1970s Germany (insert citation), 1980s Spain (insert citation), and 1990s Russia (insert citation). One the other hand, other studies do not find such an effect analyzing other Central and Eastern European countries (insert citation). Furthermore, more recent studies even find a decline in PSD over the last years (insert citation).

Regarding the second theory, instrumental regime performance, PSD rises if the system performs well in terms of instrumental benefits such as economic growth and it declines if the opposite is the case (insert citation). Consequently, the theory suggests that during PSD declines during economic crises. However, there are several case-studies that do (insert citation) and that do not (insert citation) find this relationship in the data.

Claassen (2020b) offers an alternative theory in transferring the thermostatic model of public opinion and policy (insert citation) to democracy and democratic support. In particular, he suggests that there is a negative feedback loop between PSD and democracy so that PSD decreases if democracy supply increases and vice versa. In short, the reasoning is that the output of democratic rights and institutions overshoots the initial desire for them which causes another overcompensation in favor of lower levels of democracy. Moreover, Claassen (2020b) finds different causal channels for electoral and minoritarian democracies. These two sub-types differ in the degree to which the majority holds juridical power. Therefore, he tests the following hypotheses: First, increases in democracy have a negative effect on PSD (H1), Second, he specifically looks at electoral (H1-elec) and minoritarian democracies (H1-min). In his analysis he finds evidence for H1 and H1-min but not for H1-elec.

\hypertarget{democratic-support-and-survival-of-democracy}{%
\subsection{Democratic Support and Survival of Democracy}\label{democratic-support-and-survival-of-democracy}}

Claassen (2020a) distinguishes between two types of PSD: diffuse and specific. Specific PSD is instrumental and focuses on regime outputs (similar to the instrumental regime performance theory), whereas diffuse support is normative and focuses on the principles of the regime. Therefore, principled PSD is more durable than specific support and helps cushioning regimes in times of political or economic crises. It is thus principled PSD that helps to ensure the survival of democracy. Although the theory is widely accepted (insert citations), the findings so far have also been mixed with supporting contributions (insert citations) and contributions against the theory (insert citations). Astonishingly, these studies essentially analyze the same data from the World Value Survey starting at wave 3 where respective PSD items are included.

Claassen (2020a) does not only look at the relationship between PSD and democractic survival but also between PSD and democratic emergence. He includes the argument made by Qi and Shin (2011) (insert citation) that PSD may also function as democratic demand, thus increasing the probability of transitioning from autocracy to democracy. Claassens hypotheses thus are: First, PSD is positively associated with subsequent change in democracy regardless of the initial level of democracy. Second, he specifically looks at PSD in already-existing democracies (H2-dem) and at PSD in autocracies (H2-aut). He finds evidence in support of H2-dem, mixed evidence for H2 and no evidence for H2-aut.

\hypertarget{the-data-on-democratic-support}{%
\subsection{The Data on Democratic Support}\label{the-data-on-democratic-support}}

Both publications use panel data for PSD in many countries and measured by multiple items. However, the respective datasets display many gaps. To solve this problem, Claassen describes a method to estimate the data in the first publication of the three-part series, Claassen (2019).

\hypertarget{model-and-estimation}{%
\section{Model and Estimation}\label{model-and-estimation}}

Explain Method in 2018b

\hypertarget{results-and-discussion}{%
\section{Results and Discussion}\label{results-and-discussion}}

\hypertarget{conclusion}{%
\section{Conclusion}\label{conclusion}}

\hypertarget{appendix-mathematical-background}{%
\section{Appendix: Mathematical Background}\label{appendix-mathematical-background}}

\hypertarget{introduction-1}{%
\subsection{Introduction}\label{introduction-1}}

The Metropolis-Hastings (MH) algorithm is a method for sampling data points from a probability distribution from which direct sampling is difficult. It places among the top 10 algorithms with the greatest influence on science and engineering in the 20th century (Beichl and Sullivan 2000). The MH algorithm belongs to the class of Markov chain Monte Carlo (MCMC) methods. In my explanation I assume prior knowledge on Monte Carlo sampling. However, I will describe the basics of Markov Chains. This section is structured as follows. First, I motivate the usage of the MH algorithm. Second, I explain the basics of Markov Chains. Third, I derive the algorithm and make clear why it works.

\hypertarget{motivation}{%
\subsection{Motivation}\label{motivation}}

One main application for the MH algorithm is Bayesian inference. Specifically, we want to estimate parameters \(\theta\) of some probabilistic model \(f\). We have only limited prior knowledge of the distribution of \(\theta\), \(p(\theta)\), and we have a likelihood sample of \(f\) given the unknown \(\theta\), namely \(p(X|\theta)\). The goal is to estimate the posterior distribution of \(\theta\), \(p(\theta|X)\), given all information that we have. In practice, we do not have a formal definition of the likelihood but only observations. Therefore, we can only approximate the posterior by numerical integration, i.e., we need to sample many points from the posterior to describe it. We can then use the posterior sample to estimate \(\theta\) with the maximum a posteriori probability estimate.\newline

\noindent
Claassen (2019) uses the MH algorithm to impute gaps in a panel data set. His approach consists of four steps: First, assume a data generating process \(f\) parameterized by \(\theta\). Second, provide the algorithm with the incomplete data \(X\) as likelihood and select priors for \(\theta\) to obtain the posterior distribution \(p(\theta|X) = \frac{p(X|\theta)p(\theta)}{P(X)} \propto p(X|\theta)p(\theta)\). Third, use the values for \(\theta\) with the highest posterior probability as estimates for \(\theta\). Finally, insert these estimates into the assumed probabilistic, data generating model \(f\) and use it to sample the missing data.\newline

\noindent
In general and abstracting from Bayesian inference, the MH algorithm generates a sample of random states according to the desired probability distribution \(P(X)\). For this purpose, the algorithm employs a Markov process that converges to a unique stationary distribution \(\pi(x)\) with \(\pi(x)=P(X)\). This distribution can then be used for further steps as previously described. The next section explains the conceptual basics.

\hypertarget{markov-chains}{%
\subsection{Markov Chains}\label{markov-chains}}

A Markov chain \((X_t)_{t \in \mathbb{N}}\) is a stochastic process (over time) with the property that the probability of the realization in the next period depends solely on the realization in the current state and not the complete history. This is called the Markov property. Because Markov chains with a countable, or discrete, state space are much more accessible than their continuous variant, in this chapter we will look at the discrete case. Formally, the Markov property writes

\begin{equation}
\label{eq:markov-property}
P(X_{t+1} |X_{t}, X_{t-1}, ..., X_{0}) = P(X_{t+1} |X_{t}).
\end{equation}

\noindent
Under some conditions, the stochastic process described by a Markov chain converges to a time-invariant probability distribution, i.e.~\(P(X_{t+k} |X_{t+k-1}) = P(X_{t} |X_{t-1}), \forall k>0\). The crucial step for understanding the MH is to see how it samples a Markov Chain that is certain to converge to a stable posterior distribution. Before exploring how the MH algorithm achieves this result, however, it is necessary to understand its conditions conceptually. To this end, we will use the example depicted by the following graph in Figure 1 that shows the intertemporal transition probabilities between three states representing random events.

\begin{figure}[H]
\label{fig:ex1}


\centering

\begin{tikzpicture}[->,shorten >=1pt,auto,node distance=4cm,
                thick,main node/.style={circle,draw,font=\Large\bfseries}]

  \node[main node] (2) {2};
  \node[main node] (1) [below left of=2] {1};
  \node[main node] (3) [below right of=2] {3};

  \path
    (2) edge [loop above] node {0.1} (2)
        edge [bend left] node {0.9} (3)
    (1) edge [bend left] node {1} (2)
    (3) edge [bend left] node {0.6} (1)
        edge [bend left] node {0.4} (2);      
\end{tikzpicture}

\caption{Transition Graph for Markov Chain with 3 states.}
\end{figure}

\noindent
This transition graph can be summarized by the \(n \times n\) transition matrix T where each element \((i,j)\) represents the probability of moving from state \(i\) in period \(t\) to state \(k\) in period \(t+1\), and where \(n\) represents the number of states, i.e \(T_{i,j} = P(X_{t+1}=j | X_t = i)\). For our example, we have

\begin{equation}
\label{eq:transition-matrix}
T=
\begin{pmatrix}
0 & 1 & 0\\
0 & 0.1 & 0.9\\
0.6 & 0.4 & 0
\end{pmatrix}
.
\end{equation}

\hypertarget{limit-distribution}{%
\subsubsection{Limit Distribution}\label{limit-distribution}}

As touched upon in the previous subsection, interesting questions can be what the probabilities of each state \(j \in \{1, ..., s\}\) are after a finite number or infinitely many steps. For this purpose let \(\pi_t (j) = P(X_t = j)\) denote the probability of being in state \(j\) in period \(t\). Of course, the probabilities in \(t>0\) depend on the probabilities for the the initial state \(\pi_0\). We can use the law of total probability to calculate the probability of each state for the next period \(t=1\) by

\begin{equation}
\label{eq:tot-prob}
P(X_1 = j) = \sum_{i=1}^{3} P(X_1 = j | X_0 = i) \pi_0(i).
\end{equation}

\noindent
I.e., to compute the probability of being in state \(j\) in \(t=1\), for each initial state \(i\), we multiply its probability \(\pi_0(i)\) by the probability of moving from \(i\) to state \(j\). This is equivalent to \(\pi_1 = \pi_0 T\) in vector notation. Further, we can compute the distributions in an arbitrary future period by repeating the matrix multiplication, e.g, \(\pi_2 = \pi_0 T T\), or in general, \(\pi_t = \pi_0 T^t\).

Now we are ready to define the limit distribution that describes the probability distribution after infinitely many periods by

\begin{equation}
\label{eq:lim-dist}
\pi_{\infty} = lim_{t \rightarrow \infty} \pi_t = lim_{t \rightarrow \infty} \pi_0 T^t.
\end{equation}

\noindent
We can further ask two additional important questions. First, does a limit distribution exist? And second, is it unique, or in other word, do we have the same limit distribution independent from the realization of the initial state \(X_0\)? In our example, there does not only exist a limit distribution with \(\pi_{\infty} = (0.2, 0.4, 0.4)\), it is even unique regardless of start distribution \(\pi_0\). This means that independent of the start state, the probability of each state converges to the same number. For the context of the MH algorithm, this is an important property because we always want to compute the same estimates for our parameters \(\theta\), regardless of the starting values of our simulation. In the next section, we introduce and simplify conditions that guarantee a unique limit distribution.

\hypertarget{irreducibility-periodicity-and-stationarity}{%
\subsubsection{Irreducibility, Periodicity and Stationarity}\label{irreducibility-periodicity-and-stationarity}}

\begin{definition}
A Markov chain is called \textit{irreducible} if each state is reachable from any other state in a finite number of steps.
\end{definition}

\noindent
Figure 2 shows a Markov chain represented by a bipartite graph. This graph is composed by two times the graph in Figure 1. Obviously, this chain is not irreducible because the initial state impacts all future distributions. More precisely, starting in one subgraph sets the probability of reaching states in the other subgraph to zero. We see that a Markov Chain is only irreducible if there is at least an indirect link between every pair of states. We also observe that if the Markov Chain is not irreducible there can be no limit distribution.

\begin{figure}[H]
\label{fig:ex2}
\centering

\begin{tikzpicture}[->,shorten >=1pt,auto,node distance=3cm,thick,main node/.style={circle,draw,font=\Large\bfseries}]

  \node[main node] (2) {2};
  \node[main node] (1) [below left of=2] {1};
  \node[main node] (3) [below right of=2] {3};
  \node[main node] (4) [right of=3]{4};
  \node[main node] (5) [above right of=4] {5};
  \node[main node] (6) [below right of=5] {6};


  \path
    (2) edge [loop above] node {0.1} (2)
        edge [bend left] node {0.9} (3)
    (1) edge [bend left] node {1} (2)
    (3) edge [bend left] node {0.6} (1)
        edge [bend left] node {0.4} (2)
    (5) edge [loop above] node {0.1} (5)
        edge [bend left] node {0.9} (6)
    (4) edge [bend left] node {1} (5)
    (6) edge [bend left] node {0.6} (4)
        edge [bend left] node {0.4} (5); 
\end{tikzpicture}

\caption{Transition Graph for Irreducible Markov Chain.}

\end{figure}

\begin{definition}
A state $i$ has a period $k$ if the greatest common denominator of possible revisits is $k$. A Markov chain is \textit{aperiodic} if the period of all its states is 1.
\end{definition}

\noindent
Consider the five-state Markov chain in Figure 3 as an illustration for the above definition and suppose we start in state 1. Observe that, independent of the random draw for next period, we will arrive again in state 1 after two or four steps. Therefore, state 1 has a period of 2. If a state is revisited in random rather than a fixed time period then the state has period 1. This is automatically the case if a state has a positive edge with itself.

\begin{figure}[H]
\label{fig:ex3}
\centering

\begin{tikzpicture}[->,shorten >=1pt,auto,node distance=3cm,thick,main node/.style={circle,draw,font=\Large\bfseries}]
  
  
    \node[main node] (1) {1}; 
    \node[main node] (2) [right of=1] {2};
    \node[main node] (3) [below of=1] {3};
    \node[main node] (4) [left of=1] {4};   
    \node[main node] (5) [left of=4] {5};   


   \path
    (1) edge [bend left] node {1/3} (2)
    (1) edge [bend left] node[right] {1/3} (3)
    (1) edge [bend left] node {1/3} (4)

    
    (2) edge [bend left] node {1} (1)
    
    (3) edge [bend left] node {1} (1)
    
    (4) edge [bend left] node {1} (5)
    (5) edge [bend left] node {1} (4)
    (4) edge [bend left] node {1} (1);



  \end{tikzpicture}
  
  \caption{Markov Chain with 2-periodic State 1}

\end{figure}

\begin{definition}
$\pi^*$ is the \textit{stationary distribution} of a Markov Chain with Transition matrix T if $\pi^* = \pi^* T$ and $\pi^*$ is a probability vector.
\end{definition}

\noindent
Verbally, this means that the probability distribution \(\pi^*\) does not change anymore over time. If \(\pi^*\) is also unique, then \(\pi^*\) is our aim, the limit distribution introduces in section 1.3.1, i.e., \(\pi^*=\pi_{\infty}\).

These three definitions are enough to understand the next fundamental theorem.

\hypertarget{the-fundamental-theorem-of-markov-chains}{%
\subsubsection{The Fundamental Theorem of Markov Chains}\label{the-fundamental-theorem-of-markov-chains}}

The next theorem defines formally the condition when a Markov Chain converges to a unique distribution, i.e.~the limit distribution.

\begin{theorem} (Fundamental Theorem of Markov Chains)
If a Markov chain is irreducible and aperiodic (called ergodic) then it has a stationary distribution $\pi^*$ that is unique ($\lim_{t \rightarrow \infty} P(X_t = i) = \pi_i^*, \forall i$).
\end{theorem}

\noindent
Therefore, if we want to construct a stable distribution \(P(X)\) via Markov chains, we need to ensure that it is irreducible and aperiodic with stationary distribution \(\pi^*=P(X)\). In the next subsection, we substitute the stationarity condition by a stronger one before we finally derive the MH algorithm.

\hypertarget{reversibility}{%
\subsubsection{Reversibility}\label{reversibility}}

\begin{definition}
A Markov chain is \textit{reversible} if there is a probability distribution $\pi$ over its states such that $\pi(i) T_{ij} = \pi(j)T_{j,i}, \forall i,j$ (reversibility condition).
\end{definition}

\begin{theorem}
A sufficient condition for distribution $\pi^*$ to be a stationary distribution of a Markov chain with transition matrix T is that it fullfills the reversibility condition.
\end{theorem}

\begin{proof}
$\sum_i \pi(i) T_{i,j} = \sum_i \pi(j) T_{j,i} = \pi(j) \sum_i  T_{j,i} = \pi(j) \implies \pi T = \pi$
\end{proof}

\noindent
Reversibility is a stronger condition than stationarity because it requires that the probability flux from \(i\) to \(j\) is equal to the one from \(j\) to \(i\) for each possible pair of states. Recall, that stationarity only requires that the probability flux to one state is equal on aggregate and not that it is symmetric between each pair of states over time. Therefore, if we want to achieve a stationary distribution it is enough to ensure that it is reversible.

\hypertarget{the-algorithm}{%
\subsection{The Algorithm}\label{the-algorithm}}

Recall that we want to generate a sample of a desired distribution \(P(X)\). For
this purpose, we use a Markov process that is uniquely defined by its transition probabilities
\(P(X_{t+1}|X)\) with limit distribution \(\pi\) so that \(\pi=P(X)\). As explained in the previous section, a Markov process has a limit distribution if each transition \(X_t \rightarrow X_{t+1}\) is reversible and if the stationary distribution \(\pi\) is ergodic. With the MH algorithm, we construct such a Markov process with stationary distribution \(\pi=P(X)\). The derivation starts
with another way of writing reversibility\footnote{We simplify our notation by using \(x'\) and \(x\) instead of \(X_{t+1}\) and \(X_t\).}:

\begin{equation}
P(x'|x)P(x) = P(x|x')P(x') \iff \frac{P(x'|x)}{P(x|x')} = \frac{P(x')}{P(x)}
\label{eq:trans}
\end{equation}

\noindent
The main idea is to separate transition \(P(x'|x)\) in two steps: the proposal step
and the acceptance-or-rejection step. Let \(g(x')\) be the proposal distribution, i.e.,
the conditional probability of proposing state \(x'\) given \(x\). And let \(A(x'|x)\) be the probability of accepting proposed state \(X'\). Formally, we have
\(P(x'|x)=g(x'|x) A(x'|x')\). Inserting this in Equation \eqref{eq:trans} gives

\begin{equation}
\frac{P(x')}{P(x)} = \frac{g(x'|x)A(x',x)}{g(x|x')A(x',x)} \iff \frac{A(x',x)}{A(x,x')} = \frac{P(x')}{P(x)}\frac{g(x|x')}{g(x'|x)}.
\label{eq:two-steps}
\end{equation}

\noindent
The following choice, termed the Metropolis choice, is commonly used as an acceptance ratio for sampling \(x'\) from \(P(x')\) that fulfills the above reversibility condition:

\begin{equation}
A(x',x) = \text{min}\left( 1, \frac{P(x')}{P(x)}\frac{g(x|x')}{g(x'|x)} \right)
\label{eq:Metropolis-choice}
\end{equation}

\noindent
Note that the minimizer in \(A(x',x)\) enforces that the probability is below 1. The MH algorithm writes as follows:

\begin{algorithm}[H]
\caption{Metropolis-Hastings algorithm}
\begin{algorithmic}
\State {Initialize $X_0$}

        \For{$t \gets 0$ to $T-1$} 
          \State {Draw $u \sim \mathcal{U}_{[0,1]}$}
          \State {Draw candidate $X^* \sim P(X^*|X_{t-1})$}
          \If{$u < \text{min}\{1, \frac{p(X^*)g(X_t|X^*)}{p(X_t)g(X^*|X_t)}\}$} 
              \State $X_{t+1} \gets X^*$
          \Else
              \State $X_{t+1} \gets X_t$
\EndIf 
        \EndFor   
\end{algorithmic}
\end{algorithm}

\noindent
Obviously, the construction of the acceptance ratio ensures reversibility. Ergodicity is ensured by the random nature with which we accept proposed states: First, the chain is irreducible because each state is reachable from any other state with positive probability at every single step. Second, for each state \(x\), \(P(x'=x)\) is always positive and therefore the Markov chain is aperiodic.

In a general setting, the choice for transition distribution \(g(x'|x)\) and the number of iterations until the limit distribution is reached are unclear. These two choices are the hyperparameters of the MH algorithm. In the Bayesian inference application in the article series staring from Claassen (2019), additional choices are the prior distribution \(p(\theta)\) and the model choice \(f\).

\newpage

\hypertarget{references}{%
\section*{References}\label{references}}
\addcontentsline{toc}{section}{References}

\singlespacing

\hypertarget{refs}{}
\begin{CSLReferences}{1}{0}
\leavevmode\vadjust pre{\hypertarget{ref-beichl2000metropolis}{}}%
Beichl, Isabel, and Francis Sullivan. 2000. {``The Metropolis Algorithm.''} \emph{Computing in Science \& Engineering} 2(1): 65--69.

\leavevmode\vadjust pre{\hypertarget{ref-Claassen2019estimating}{}}%
Claassen, Christopher. 2019. {``Estimating Smooth Country--Year Panels of Public Opinion.''} \emph{Political Analysis} 27(1): 1--20.

\leavevmode\vadjust pre{\hypertarget{ref-Claassen2020support}{}}%
---------. 2020a. {``Does Public Support Help Democracy Survive?''} \emph{American Journal of Political Science} 64(1): 118--34.

\leavevmode\vadjust pre{\hypertarget{ref-Claassen2020mood}{}}%
---------. 2020b. {``In the Mood for Democracy? Democratic Support as Thermostatic Opinion.''} \emph{American Political Science Review} 114(1): 36--53.

\leavevmode\vadjust pre{\hypertarget{ref-Denemark2016}{}}%
Denemark, David, Todd Donovan, Richard G Niemi, and Robert Mattes. 2016. {``The Advanced Democracies: The Erosion of Traditional Democratic Citizenship.''} \emph{Growing up democratic: Does it make a difference}: 181--206.

\leavevmode\vadjust pre{\hypertarget{ref-Foa2016}{}}%
Foa, Roberto Stefan, and Yascha Mounk. 2016. {``The Danger of Deconsolidation: The Democratic Disconnect.''} \emph{Journal of democracy} 27(3): 5--17.

\leavevmode\vadjust pre{\hypertarget{ref-Foa2017}{}}%
---------. 2017. {``The Signs of Deconsolidation.''} \emph{Journal of democracy} 28(1): 5--15.

\leavevmode\vadjust pre{\hypertarget{ref-Norris2017}{}}%
Norris, Pippa. 2017. {``Is Western Democracy Backsliding? Diagnosing the Risks.''} \emph{Forthcoming, The Journal of Democracy, April}.

\leavevmode\vadjust pre{\hypertarget{ref-Voeten2016}{}}%
Voeten, Erik. 2016. {``Are People Really Turning Away from Democracy?''} \emph{Available at SSRN 2882878}.

\end{CSLReferences}

\clearpage

\hypertarget{statutory-declaration}{%
\section*{Statutory Declaration}\label{statutory-declaration}}
\addcontentsline{toc}{section}{Statutory Declaration}

Hiermit versichere ich, dass diese Arbeit von mir persönlich verfasst ist und dass ich keinerlei fremde Hilfe in Anspruch genommen habe.
Ebenso versichere ich, dass diese Arbeit oder Teile daraus weder von mir selbst noch von anderen als Leistungsnachweise andernorts eingereicht wurden.
Wörtliche oder sinngemäße Übernahmen aus anderen Schriften und Veröffentlichungen in gedruckter oder elektronischer Form sind gekennzeichnet.
Sämtliche Sekundärliteratur und sonstige Quellen sind nachgewiesen und in der Bibliographie aufgeführt.
Das Gleiche gilt für graphische Darstellungen und Bilder sowie für alle Internet-Quellen.
Ich bin ferner damit einverstanden, dass meine Arbeit zum Zwecke eines Plagiatsabgleichs in elektronischer Form anonymisiert versendet und gespeichert werden kann.
Mir ist bekannt, dass von der Korrektur der Arbeit abgesehen werden kann, wenn die Erklärung nicht erteilt wird.

\SignatureAndDate{}
\renewcommand*{\thepage}{ }

\noindent I hereby declare that the paper presented is my own work and that I have not called upon the help of a third party.
In addition, I affirm that neither I nor anybody else has submitted this paper or parts of it to obtain credits elsewhere before.
I have clearly marked and acknowledged all quotations or references that have been taken from the works of other.
All secondary literature and other sources are marked and listed in the bibliography.
The same applies to all charts, diagrams and illustrations as well as to all Internet sources.
Moreover, I consent to my paper being electronically stores and sent anonymously in order to be checked for plagiarism.
I am aware that the paper cannot be evaluated and may be graded \enquote{failed} (\enquote{nicht ausreichend}) if the declaration is not made.

\SignatureAndDateEng{}

% % % 
\end{document}
